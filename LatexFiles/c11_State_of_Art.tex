El objetivo de la Vision artificial, es la recolecci\'on de informaci\'on a partir de diferentes fuentes de im\'agenes y videos. Dentro de este campo podemos encontrarnos la seguridad y vigilancia \cite{traffic_surveillance_pergamon} \cite{distributed_surveillance} \cite{vehicle_detection}, reconocimiento facial, an\'alisis deportivo o incluso ocio y juegos (como el popular Kinect de Microsoft \cite{Kinect_intro})

Este projecto, se centra en el uso de la visi\'on artificial para extraer informaci\'on sobre la posicion de objetos para realizar acciones de vigilancia. La distribuci\'on de tareas \cite{Coop_Surv_aerial_JJ} \cite{Consensus_reaching_Xiao} \cite{Adaptative_tast_Meuth} \cite{distributed_architecture_Ivan_Maza} es una pieza clave en este campo, ya que permite un uso efectivo de los activos disponibles. La descentralizaci\'on de tare \cite{descentralized_task_UAV} conlleva la necesidad del uso de redes que permiten a los diferences dispositivos comunicarse entre ellos y compartir la informaci\'on.

En particular, respecto a los algoritmos de visi\'on, existen n\'umerosos m\'etodos de extracci\'on de informaci\'on. Algunos centrados en el reconocimiento de formas \cite{shape_using_shape_context} \cite{Vehicle_recog_markov}, segmentaci\'on por color, o una combinaci\'on de estos \cite{realtime_signal_recon_shape_color} \cite{signal_recogn_shape_color} \cite{Robust_RT_tracking_color_face_Darrell}.


% 666 TODO....